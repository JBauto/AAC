\chapter{Unidade de controlo de saltos e flags}
\textsc{Breve descrição da unidade: } Desenhou-se a unidade de modo a controlar o próximo endereço a enviar ao \textit{Program Counter} (\textbf{PC}).\\
Esta unidade guarda os valores das \textit{flags} provenientes da \textbf{ALU} em registos e depois usa esses registos para calcular as condições de salto.\par
Juntámos os registos das \textit{flags} com a unidade de controlo de saltos de modo a que consoante a condição de salto vinda do \textbf{Decoder} se pudesse calcular se se deveria executar um salto ou se permitíamos que o valor do \textbf{PC} fosse incrementado normalmente.\par
O cálculo do próximo endereço do \textbf{PC} é feito em 4 fases.
\begin{enumerate}
	\setlength{\itemindent}{25pt}
	\item Cálculo da condição de salto
	\item Cálculo do \textit{offset} para o caso de saltos no Formato I ou do Formato II
	\item Determinar se o salto é para um \textit{offset} ou para um registo
	\item Determinar o próximo endereço do \textbf{PC} com base no tipo de \textit{jump} (condicional ou incondicional) e a condição
\end{enumerate}

\[\text{Condição}=\left\{
	\begin{array}{rcl}
	1 & : & COND=0000\\
	flagV & : & COND=0011\\
	flagS & : & COND=0100\\
	flagZ & : & COND=0101\\
	flagC & : &  COND=0110\\
	flagS+flagZ & : & COND=0111\\
	0 & : & others
	\end{array}\right.\]

\[\text{Offset}=\left\{
	\begin{array}{rcl}
	\begin{array}{r}
	Destino(11)\&Destino(11)\&Destino(11)\&Destino(11)\\
	\&Destino(11\ downto\ 0)
	\end{array} & : & OP=10\\
	\begin{array}{r}
	Destino(7)\&Destino(7)\&Destino(7)\&Destino(7)\&Destino(7)\\
	\&Destino(7)\&Destino(7)\&Destino(7)\&Destino(7\ downto\ 0)
	\end{array} & : & others
	\end{array}\right.\]

\[\text{Jump Address}=\left\{
	\begin{array}{rcl}
	RB & : & OP=11\\
	PC+1+Offset & : & others
	\end{array}\right.\]

\[\text{Próximo PC}=\left\{
	\begin{array}{rcl}
	Jump\ Address & : & enable\_jump=1\ \cdot\ (\text{Condição} \oplus OP(0))=1\\
	Jump\ Address & : & enable\_jump=1\ \cdot\ OP(1)=1\\
	PC+1 & : & others
	\end{array}\right.\]
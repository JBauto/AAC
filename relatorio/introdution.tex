\chapter{Introdução}
O objectivo deste terceiro trabalho de laboratório é a aceleração de um algoritmo de \textit{smoothing} utilizando as propriedades de computação paralela em GPUs.

Para tal recorreu-se à plataforma CUDA\texttrademark  que tira proveito das unidades de processamento gráfico (GPUs) da NVIDIA\texttrademark.

Procura-se tirar proveito da arquitectura dos GPUs para maximizar o desempenho de um algoritmo de \textit{smoothing} recorrendo ao Paralelismo de Dados.

\section{Algoritmo}

O algoritmo de \textit{smoothing} é um algoritmo bastante utilizado no processamento de imagem bem como análise estatística com o objectivo de criar uma função aproximada dos dados de entrada mantendo pontos importantes nos dados enquanto que reduz qualquer tipo de ruído associado aos dados. 

Para modelar os dados o algoritmo observa um ponto individual bem como os imediatamente adjacentes e caso o ponto a observar seja maior em valor que os seus adjacentes é suavizado diminuindo o seu valor. Se o valor for menor que os seus adjacentes é entre elevado o seu valor em relação à sua volta. Isto assume que estas elevações bruscas de valores deve-se a ruído no sinal.

O algoritmo que pretendemos paralelizar consiste em:
\[ \hat{y}_i=\frac{\sum_{k=0}^{N-1} K_{b}(x_i,x_k)y_k}{\sum_{k=0}^{N-1} K_{b}(x_i,x_k)} \]
com
\[ K_b(x,x_k)=exp\left ( -\frac{\left ( x-x_k \right )^2}{2b^2} \right ) \]
onde
\begin{description}
	\item[x] \hfill \\
	Domínio do sinal a ser filtrado
	\item[y] \hfill \\
	Sinal observado e que contém ruído e do qual pretendemos obter a versão sem ruído
	\item[$\hat{y}$] \hfill \\
	Sinal obtido pela passagem do sinal y pela função de \textit{smoothing}
	\item[b] \hfill \\
	Parâmetro de \textit{smoothing}, no nosso caso foi utilizado o valor 4 para este parâmetro
\end{description}
\chapter{Unidade de Constantes}
\textsc{Breve descrição da unidade: } A unidade de Constantes é responsável pelas operações apresentadas na tabela~\ref{tabela:constantes}.
\begin{table}[h]
	\centering
	\begin{tabular}{|c|c|c|}
		\hline
		Formato & Operação & Mnemónica \\ \hline
		I & \mbox{$C=Constante$} & loadlit c, Const  \\ \hline
		II & \mbox{$C=Const8|(C\&0xff00)$}  & lcl c, Const8 \\ \hline
		II & \mbox{$C=(Const8<<8)|(C\&0x00ff)$}  & lch c, Const8 \\ \hline
	\end{tabular}
	\caption{Operações com Constantes}
	\label{tabela:constantes}
\end{table}

Optámos por separar estas operações das restantes da \textbf{ALU} de modo a facilitar a descodificação das instruções por parte do \textit{decoder}.
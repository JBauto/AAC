\documentclass[11pt,a4paper,titlepage,onecolumn]{report}

% Corrigir margens e paragrafos
\usepackage[hmarginratio=1:1]{geometry}
\usepackage{parskip}
\usepackage{mathtools}
\usepackage{amsfonts}
\usepackage[titletoc]{appendix}
\usepackage{float}
\usepackage{pdflscape}

% Definir linguagem portuguesa
\usepackage[utf8]{inputenc}
\usepackage[T1]{fontenc}
\usepackage[portuguese]{babel}

\setlength{\textheight}{21cm}
\setlength{\voffset}{-0.5cm}

% Incluir packages de cores
\usepackage{graphicx}
\usepackage{color}
\usepackage{xcolor}


\usepackage{indentfirst}
\parindent=18pt

\usepackage{chngcntr}

\usepackage[T1]{fontenc}
\usepackage{titlesec, blindtext, color}
\definecolor{gray75}{gray}{0.75}
\newcommand{\hsp}{\hspace{20pt}}
\titleformat{\chapter}[hang]{\Large\bfseries}{\chaptername \ \thechapter\hsp\textcolor{gray75}{|}\hsp}{0pt}{\LARGE\bfseries}


%subfigure
\usepackage[hang,small,bf]{subfigure} 


\begin{document}
	\chapter{Unidade lógico-aritmética (ALU)}
	\textsc{Breve descrição da \textbf{ALU}: } Desenhou-se a \textbf{ALU} com três unidades a funcionarem em paralelo, abaixo descritas com maior~detalhe.\\
	O resultado produzido por estas unidades é introduzido num ``multiplexer'' que escolhe de acordo com sinais provenientes da unidade de descodificação qual o resultado e ``Flags'' a colocar à saída da~\textbf{ALU}.
	
	\section{Unidade Aritmética}
	A unidade Aritmética é responsável pelas operações apresentadas na tabela \ref{tabela:arith}.\\
		
	\begin{table}[h]
		\centering
		\begin{tabular}{|c|c|c|c|}
			\hline
			OP    & Operação & Mnemónica & Flags actualizadas \\ \hline
			00000 & \mbox{$C=A+B$}    & add c, a, b    & S,C,Z,V   \\ \hline
			00001 & \mbox{$C=A+B+1$}  & addinc c, a, b & S,C,Z,V   \\ \hline
			00011 & \mbox{$C=A+1$}    & inca c, a      & S,C,Z,V   \\ \hline
			00100 & \mbox{$C=A-B-1$}  & subdec c, a, b & S,C,Z,V   \\ \hline
			00101 & \mbox{$C=A-B$}    & sub c, a, b    & S,C,Z,V   \\ \hline
			00110 & \mbox{$C=A-1$}    & deca c, a      & S,C,Z,V   \\ \hline
		\end{tabular}
		\caption{Operações aritméticas}
		\label{tabela:arith}
	\end{table}
	
	A unidade aritmética começa por analisar qual a operação a executar de acordo com os dados vindos da unidade de descodificação e em seguida começa por calcular o segundo membro da operação \mbox{$C=A+operB$} em que 
	\[ operB=\left\{
		\begin{array}{lr}
		B & : OP=00000\\
		B+1 & : OP=00001\\
		1 & : OP=00011\\
		-B-1 & : OP=00100\\
		-B & : OP=00101\\
		-1 & : OP=00110
		\end{array}
		\right.\]
	De seguida calcula \mbox{$C=A+operB$} e as ``Flags'' correspondentes com base na análise do resultado e dos operandos.
	
	\section{Unidade Lógica}
	
	\section{Unidade de Deslocamentos}
	
\end{document}
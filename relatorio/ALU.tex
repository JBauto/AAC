\documentclass[11pt,a4paper,titlepage,onecolumn]{report}

% Corrigir margens e paragrafos
\usepackage[hmarginratio=1:1]{geometry}
\usepackage{parskip}
\usepackage{mathtools}
\usepackage{amsfonts}
\usepackage[titletoc]{appendix}
\usepackage{float}
\usepackage{pdflscape}

% Definir linguagem portuguesa
\usepackage[utf8]{inputenc}
\usepackage[T1]{fontenc}
\usepackage[portuguese]{babel}

\setlength{\textheight}{21cm}
\setlength{\voffset}{-0.5cm}

% Incluir packages de cores
\usepackage{graphicx}
\usepackage{color}
\usepackage{xcolor}


\usepackage{indentfirst}
\parindent=18pt

\usepackage{chngcntr}

\usepackage[T1]{fontenc}
\usepackage{titlesec, blindtext, color}
\definecolor{gray75}{gray}{0.75}
\newcommand{\hsp}{\hspace{20pt}}
\titleformat{\chapter}[hang]{\Large\bfseries}{\chaptername \ \thechapter\hsp\textcolor{gray75}{|}\hsp}{0pt}{\LARGE\bfseries}


%subfigure
\usepackage[hang,small,bf]{subfigure} 


\begin{document}
	\chapter{Unidade lógico-aritmética (ALU)}
	\textsc{Breve descrição da \textbf{ALU}: } Desenhou-se a \textbf{ALU} com três unidades a funcionarem em paralelo, abaixo descritas com maior~detalhe.\\
	O resultado produzido por estas unidades é introduzido num ``multiplexer'' que escolhe de acordo com sinais provenientes da unidade de descodificação qual o resultado e ``Flags'' a colocar à saída da~\textbf{ALU}.
	
	\section{Unidade Aritmética}
	A unidade Aritmética é responsável pelas operações apresentadas na tabela~\ref{tabela:arith}.
		
	\begin{table}[h]
		\centering
		\begin{tabular}{|c|c|c|c|}
			\hline
			OP    & Operação & Mnemónica & Flags actualizadas \\ \hline
			00000 & \mbox{$C=A+B$}    & add c, a, b    & S,C,Z,V   \\ \hline
			00001 & \mbox{$C=A+B+1$}  & addinc c, a, b & S,C,Z,V   \\ \hline
			00011 & \mbox{$C=A+1$}    & inca c, a      & S,C,Z,V   \\ \hline
			00100 & \mbox{$C=A-B-1$}  & subdec c, a, b & S,C,Z,V   \\ \hline
			00101 & \mbox{$C=A-B$}    & sub c, a, b    & S,C,Z,V   \\ \hline
			00110 & \mbox{$C=A-1$}    & deca c, a      & S,C,Z,V   \\ \hline
		\end{tabular}
		\caption{Operações Aritméticas}
		\label{tabela:arith}
	\end{table}
	
	A unidade aritmética começa por analisar qual a operação a executar de acordo com os dados vindos da unidade de descodificação e em seguida começa por calcular o segundo membro da operação \mbox{$C=A+operB$} em que 
	\[ operB=\left\{
		\begin{array}{lr}
		B & : OP=00000\\
		B+1 & : OP=00001\\
		1 & : OP=00011\\
		-B-1 & : OP=00100\\
		-B & : OP=00101\\
		-1 & : OP=00110
		\end{array}
		\right.\]
	De seguida calcula \mbox{$C=A+operB$} e as ``Flags'' correspondentes com base na análise do resultado e dos operandos.
	
	\section{Unidade Lógica}
	
	A unidade Lógica é responsável pelas operações apresentadas na tabela~\ref{tabela:logic}.\\
	
	\begin{table}[h]
		\centering
		\begin{tabular}{|c|c|c|c|}
			\hline
			OP    & Operação & Mnemónica & Flags actualizadas \\ \hline
			10000 & \mbox{$C=0$}    & zeros c   &   \\ \hline
			10001 & \mbox{$C=A\&B$}  & and c, a, b & S,Z   \\ \hline
			10010 & \mbox{$C=!A\&B$}  & andnota c, a, b & S,Z   \\ \hline
			10011 & \mbox{$C=B$}  & passb c, b &    \\ \hline
			10100 & \mbox{$C=A\&!B$}  & andnotb c, a, b & S,Z   \\ \hline
			10101 & \mbox{$C=A$}  & passa c, a & S,Z   \\ \hline
			10110 & \mbox{$C=A \oplus B$}  & xor c, a, b & S,Z   \\ \hline
			10111 & \mbox{$C=A|B$}  & or c, a, b & S,Z   \\ \hline
			11000 & \mbox{$C=!A\&!B$}  & nor c, a, b & S,Z   \\ \hline
			11001 & \mbox{$C=!(A \oplus B)$}  & xnor c, a, b & S,Z   \\ \hline
			11010 & \mbox{$C=!A$}  & passnota c, a & S,Z   \\ \hline
			11011 & \mbox{$C=!A|B$}  & ornota c, a, b & S,Z   \\ \hline
			11100 & \mbox{$C=!B$}  & passnotb c, b & S,Z   \\ \hline
			11101 & \mbox{$C=!A|!B$}  & nand c, a, b & S,Z   \\ \hline
			11111 & \mbox{$C=1$}  & ones c &    \\ \hline
		\end{tabular}
		\caption{Operações de Deslocamento}
		\label{tabela:logic}
	\end{table}
	
	\section{Unidade de Deslocamentos}
	A unidade de Deslocamentos é responsável pelas operações apresentadas na tabela~\ref{tabela:shift}.\\
	
	\begin{table}[h]
		\centering
		\begin{tabular}{|c|c|c|c|}
			\hline
			OP    & Operação & Mnemónica & Flags actualizadas \\ \hline
			01000 & \mbox{$C=Shift Lógico Esq.(A)$}    & lsl c, a   & S,C,Z   \\ \hline
			01001 & \mbox{$C=Shift Aritmético Dir.(A)$}  & asr c, a & S,C,Z   \\ \hline
		\end{tabular}
		\caption{Operações de Deslocamento}
		\label{tabela:shift}
	\end{table}
	
	No caso do ``shift'' lógico a saída resulta do deslocamento do sinal de entrada uma posição e preenchimento do ``bit0'' com 0.\\
	No caso do ``shift'' aritmético a saída resulta do deslocamento do sinal de entrada uma posição e preenchimento do ``bit15'' com o ``bit15'' da entrada.\\
	
	
\end{document}
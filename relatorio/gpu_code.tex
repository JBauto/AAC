\chapter{Código para o GPU}

Para optimizar a aceleração do algoritmo de \textit{smoothing} no \textit{GPU} foram implementadas três diferentes versões cada uma explorando a arquitectura do \textit{GPU} de forma diferente.

%Para optimizar a aceleração do algoritmo de \textit{smoothing} no \textit{GPU} começámos por analisar as diversas chamadas ao GPU e concluímos que as que consomem mais tempo são a inicialização e as transferências de dados entre o \textit{Host} e o \textit{GPU} e entre o \textit{GPU} e o \textit{Host}, sendo que a execução do \textit{Kernel} em si consome uma porção quase negligenciável.
%
%Com estas informações e tendo em conta que as chamadas de inicialização são constantes e não se podem alterar procurámos primeiro optimizar as transferências de dados e só depois optimizar o \textit{Kernel}.

\section{Kernel}
\subsection{Versão A}

Numa primeira abordagem ao algoritmo, foi implementado um \textit{Kernel} que executava o \textit{for loop} interior do algoritmo em que cada \textit{thread} computava uma iteração do \textit{for loop} exterior. Desta forma eram necessárias N threads sendo N o número de iterações de cada \textit{loop}.
Esta implementação consegue tirar partido do conceito básico por trás dos \textit{GPUs} contudo não o faça da forma óptima e apresenta maus resultados ao nível da transferência de dados quando N é elevado. 
%\label{sec:kernel}
%O código do \textit{Kernel} a executar no \textit{GPU} pode ser visto na Secção~\ref{secA:kernel}, e este é chamado várias vezes de modo a calcular uma parte do vetor de resultados.
\subsection{Versão B}

Uma segunda versão permitia obter o máximo de paralelismo possível do \textit{GPU} criando uma \textit{grid} com as seguintes dimensões,

\[grid.x = \sqrt{\frac{N\cdot N}{maxThreadsPerBlock}} + 1\]
\[grid.y = \sqrt{\frac{N\cdot N}{maxThreadsPerBlock}} + 1\]

e dimensões de bloco,

\[block.x = \sqrt{maxThreadsPerBlock}\]
\[block.y = \sqrt{maxThreadsPerBlock}\]

Desta forma é garantido que existem threads suficientes para o cálculo do algoritmo em que cada thread executa o cálculo de uma exponencial e por fim N threads para o cálculo da soma de cada A e B e da divisão final.

Esta versão apresenta limitações ao nível da memória existente para alocar recursos uma vez que era necessário a alocação de um vector de N $\cdot$ N $\cdot$ 2 e uma vez que o \textit{GPU} apenas tem 2GB de memória disponível apenas é possível aplicar o algoritmo para, 

\[N = \sqrt{\frac{2048\cdot 1024\cdot 1024}{2\cdot 4}} = 16384\]

O principal factor que inviabilizou esta versão foi o tempo de tranferência dos dados do \textit{GPU} para \textit{Host} que para N igual a 10000 era de aproximadamente 0.25 segundos. Uma hipótese é que o tempo de transferência de dados não depende somente do tamanho do vector a tranferir mas também do total de dados alocados no \textit{GPU} em que para N igual 10000 correspondiam a cerca de 800 MB.

\subsection{Versão C}

Esta última versão segue o mesmo príncipio que a versão A contudo é definido um valor máximo para o tamanho dos vectores de saída, ou seja, se N for igual a 50000 o tamanho máximo do vector \textit{Yest} é 10000 e o \textit{Kernel} é executado 5 vezes. Esta versão afasta-se um pouco do conceito de paralelismo tentando minimizar o tempo de transferência de dados, tempo que se apresentou ser o de maior impacto.

Comparando a versão A e C relativamente à última afirmação, tem-se um tempo de 0.361632 segundos para transferir 80000 elementos para A enquanto que para C tem-se 0,070261 segundos. Esta alteração permite reduzir o tempo de transferência em 5.15 vezes.


\section{Transferências de Dados}
Inicialmente são transferidos os vectores de "entrada" \mbox{$(X\ e\ Y)$} necessários para os cálculos dos resultados, em seguida e como descrito na Secção~\ref{sec:kernel} o \textit{Kernel} é executado várias vezes para calcular uma porção dos resultados e após cada chamada ao \textit{Kernel} é transferido para o \textit{Host} os resultados estimados~(Yest) acabados de calcular.

Escolhemos fazer deste modo para permitir minimizar o impacto da transferência de informação do \textit{GPU} para o \textit{Host} pois era essa comunicação que ditava o desempenho do programa.

\section{Critério de aceitação de resultados}
Os resultados provenientes do \textit{GPU} são considerados como correctos se diferirem dos resultados obtidos no \textit{CPU} menos do que \mbox{$1\times10^{-6}$}.

Existe sempre uma ligeira diferença nos resultados devido ao facto de se tratarem de unidades aritméticas diferentes sem precisão infinita.